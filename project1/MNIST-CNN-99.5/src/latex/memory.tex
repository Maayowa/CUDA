\cleardoublepage
\setcounter{chapter}{4} %The counter for chapters should be set one less than 
                        %the actual chapter number, for example, {1} for 
                        %chapter 2, and {2} for chapter 3.
\setcounter{section}{5} %The counter for sections should be set to match the
                        %actual chapter number, for example, {2} for sections 
                        %in chapter, and {3} for sections in chapter 3. 
\chapter{WEBGUI Memory Usage}
\markboth{WEBGUI USERS' GUIDE}{WEBGUI MEMORY USAGE}
\pagenumbering{arabic}
\setcounter{page}{47} %The counter for page numbers must be placed here in your
                     %chapter, following the \chapter and \markboth commands.
                     %It should always be the next available right-hand page.
                     %All chapters start on new rights. It will always be
                     %an odd number.  

 
\section{Overview}
\label{sec:5-1}
When you compile and link \f{WEBGUI} to your software, upon running it shares the memory space of your software.
This section describes \f{WEBGUI}'s memory usage so that you are aware. There are five main consumers of memory.
Only one can really become significant which is displaying large 3D objects. 

\section{Basic}
Since webgui.c is a web server, the first use of memory is maintaining index.html in memory
to be ready whenever a web browser requests it. This is roughly 100,000 bytes. Next, if you call \textbf{webinit} and enable the
web browser to store parameters giving the user the ability to view and change parameters, these variables are stored. An average number
of command buttons, parameters, and options uses around 10,000 bytes. (Each parameter uses about 100 bytes and each command button 
uses about 50 bytes.) Next webgui.c maintains a few command string buffers; these buffers use a total of 25,000 bytes.

\section{2D Images}
The two large uses of memory are the display of 2D images and 3D objects. When a 2D image is displayed, webgui.c must keep a copy
of the image in memory even though it passes the data on to the web browser. If someone refreshes a web browser, then webgui.c must
serve up the image again. \f{WEBGUI} only allows 2D images to have a maximum of 255 colors, therefore each pixel consumes 1 byte in 
memory. Storing the palette doesn't use much memory. If your software displays a 1200 by 800 image (by calling \textbf{webimagedisplay}), 
then  1200 x 800 = 960,000 pixels were submitted. Thus this image uses
1 megabyte of memory. If the user viewing the web browser no longer needs the image, they can click a button labeled Free beneath the image
to free this 1 megabyte of memory in memory space.

\section{3D Objects}
\label{sec:5-2}
3D objects can require the most use of memory depending on how many polygons and lines make up the object. Whenever your software
calls \textbf{webline} thus sending a line with multiple vertices, the line is saved in webgui.c 's memory as separate line segments. 
When your software calls \textbf{webfill} thus sending a convex
polygon, the polygon is saved in webgui.c 's memory as separate triangles. Every line segment uses 32 bytes of memory and every triangle
uses 44 bytes of memory. Therefore if you send a large 3D object with one million line segments and one million triangles, then this object
will use 75 megabytes of memory. Even though webgui.c sends this data to the web browser, it keeps a copy of the object in memory in case
someone reloads a web page and a web page needs to receive the object again. At any time, a user can free this memory by clicking the
button labeled Free beneath the displayed 3D object.

Additionally webgui.c uses a large temporary block of memory for a few seconds. Every time \textbf{webline} is called,
then 32 more bytes are allocated for each line segment. Each time \textbf{webfill} is called, 44 bytes are used per triangle. 
This memory is allocated as the calls execute. The 3D
object isn't drawn until your software calls \textbf{webgldisplay}. When this function is called, then webgui.c allocates another temporary
block of memory equal in size to the combined memory usage of all the stored lines and triangles. For example, if one million lines
are submitted and one million triangles are submitted, then 75 megabytes of memory are allocated. When \textbf{webgldisplay} is called,
another 75 megabytes of memory is allocated (for 150 MB total). This additional 75 megabytes contains the data in a special format usable
by the GPU on the client's computer. This 75 MB is then transmitted to the client and then freed. (Afterward the server is only
using the original 75MB) Next, the web browser of the client's computer holds the 75 MB of data in memory for one second and then this 
data is passed into the client's GPU and the web browser frees the 75MB. The entire time
that the 3D object is visible on the client's screen, the GPU has the 75 MB in its memory and webgui.c has 75 MB in its memory.

These two 75 MB usages of memory are freed when the user clicks Free under the displayed 3D object, or when another 2D image or 3D object
is displayed in the same display pane. When a new 2D image or 3D object is sent to an already used display pane, then the data 
associated with the old image or object is freed and replaced by the new image or object.